\chapter{Conclusion}\label{ch:conclusion}

Voilà c'est tout pour moi, j'espère vous avoir aidé dans vos démarches et que vous êtes dorénavant plus motivé que jamais a affronté les épreuves si il y en a. Je pense que partir en bi diplôme est incroyable enrichissant, pas forcément d'un point de vue éducatif, mais surtout d'un point de vue humain. Cela vous apportera une ouverture d'esprit et une expérience que vous n'aurez pas sans sortir de chez vous. C'est aussi le moment le plus propice pour vous évader de la routine habituelle, car ce n'est pas lorsque votre situation sera stable après l'acquisition de votre diplôme que vous partirez! Je me répète encore, mais j'aurai vraiment été très déçu de ne pas avoir connu cette vie étudiante dans une telle université, c'est complètement différent de l'ESEO ou de tout ce que vous avez pu connaître d'autre. C'est une vie incroyable et je regrette même d'avoir passé 4 ans à l'ESEO alors que j'aurai pu commencer dès le début là bas.

J'espère aussi vous avoir donné l'envie de partir ainsi que détruis quelques idées reçues sur les coûts ou les conditions d'admission. J'invite aussi mes collègues à y ajouter leurs expériences et erreurs à ne plus commettre ainsi qu'à entretenir l'état des liens, mais aussi à corriger mes fautes d'orthographe! Beaucoup reste à faire, mais les bases sont posées.

Je vous souhaite bonne chance et n'oubliez pas que tout étudiant est joignable sur son adresse mail ESEO. Ils (ainsi que moi même) seront ouverts à toutes questions de votre part.

\begin{center}
  Auteur original: COSNEAU Alexandre\\
  Email: alexandre.cosneau@reseau.eseo.fr\\
\end{center}
