\chapter{Introduction}\label{ch:Intro}

L'intérêt de ce document est d'aider les futurs étudiants sur les démarches à suivre concernant le départ en bidiplôme. En montrant les intérêts, mais aussi en enlevant ce flou sur les prix ou les difficultés administratives et autres problèmes auxquels on ne pense pas spécialement. Chaque année c'est la même galère pour tout le monde, car l'ESEO n'a aucun document à jour nous informant des procédures (mis à part le PowerPoint périmé depuis une dizaine d'années qu'on nous présente à chaque fois) et aucun étudiant ne veut prendre un peu de son temps pour expliquer cela ce qui faciliterait grandement les choses.\\
Je me focaliserais sur le bidiplôme au Canada ainsi que Chicago, car ce sont les bidiplômes qui m'intéressaient particulièrement. Pour les autres bidiplômes, les démarches seront différentes et j'invite les anciens à compléter ce document.

Il est important de bien comprendre une chose primordiale: \\
\textbf{Vous êtes seul.} \\
Dans le sens où l'administration de l'ESEO est absolument incompétente et n'a aucune idée des démarches à suivre ou même des liens pour s'inscrire, mais surtout, ne VEUT pas s'améliorer là-dessus. Au mieux l'administration ne vous servira à rien, au pire elle vous ruinera vos chances d'accéder au bidiplôme voulu, ce qui fût le cas pour bon nombre d'entre nous ainsi que moi-même.

L'un des intérêts majeurs de ce document, c'est aussi de vous donner un véritable retour des personnes qui sont actuellement parties. C'est pour cela qu'il est accompagné d'anecdotes personnelles.

Je pense que l'entre-aide entre élèves est ce qui rend la vie associative si forte à l'ESEO et c'est à chacun de faire son effort pour qu'elle continue ainsi. Je souhaite que mon parcours du combattant ne soit pas réitéré par d'autres, car on parle bien ici d'une orientation de carrière et pas d'un simple choix de cours. Prenez toutes les mesures nécessaires pour ne pas vous faire avoir et choisir le bidiplôme qui vous convient.

Souvenez-vous que peu importe le pourquoi du comment vous avez loupé votre chance de partir, peu importe à qui la faute, au final c'est, vous qui ne partez pas. C'est pour cela que je vous dis que vous êtes seul, vous devez tout faire vous-même et ne comptez pas sur l'aide de l'ESEO, car ils s'en foutent royalement.

\section{Prérequis}\label{sec:sec1}

Il est important de se renseigner au moins un an avant le départ. Peu importe ce que vous dit l'ESEO (ils vous diront que vous avez le temps de vous pour vos démarches, etc.) consacrez une bonne partie de votre temps à bien comprendre les démarches à suivre ainsi que les possibles certifications a avoir, mais aussi de la culture ou des meurs du pays qui pourraient ne pas vous correspondre.

\section{Les certifications}\label{sec:sec2}

Pour certains bidiplômes, comme Chicago, deux certifications sont à obtenir pour pouvoir avoir un dossier admissible. D'autres comme Sherbrooke, n'en demande aucune. C'est-à-dire que seul l'administratif suffit, peu importe votre niveau en anglais ou autre.(Première bonne nouvelle pour certain(es))

\section{L'administratif}\label{sec:sec3}

L'administratif est sûrement la tâche la plus prenante en termes de temps et de patience. Tout est fait pour décourager l'étudiant dans sa démarche, mais avec du temps et du suivi sur l'état de votre dossier, on y arrive. Je cherche tout particulièrement à vous faciliter la tâche en termes de démarches à suivre.

\section{Un bidiplôme diplômant?}\label{sec:sec4}

En effet certains bidiplôme sont diplômants. C'est-à-dire que sous réserve d'avoir réussi votre année, vous pouvez demander une dérogation à l'ESEO pour ne pas y revenir et ainsi être diplômé de l'université en question ET de l'ESEO. \\
Cela vous permet d'embrayer directement sur un stage à l'étranger ou même un emploi.

\section{Je veux aller plus loin que le bidiplôme}\label{sec:sec5}

Beaucoup d'entre vous ont sûrement envie de partir, mais aucun bi diplôme proposé ne vous intéresse. Sachez que vous pouvez demander à des universités non partenaire de vous accepter dans leur cursus! Il faut cependant s'y prendre à l'avance et je ne pourrais pas vous aider sur les démarches. En revanche, je peux vous assurer que les universités sont totalement ouvertes aux étudiants internationaux et que leurs systèmes éducatifs sont beaucoup moins rigides que le nôtre. Vous y trouverez donc chaussure à votre pied.
